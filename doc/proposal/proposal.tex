\documentclass[11pt, rgb]{scrartcl}
\usepackage{themeKonstanzDBIS} % Muss immer verwendet werden (Standardpaket)
\format{a4}

% Thesis information        %
\date{\today}
\year{2020}
\author{Fabian Klopfer}

\title{Minimization of Stochastic Dynamical Systems}
\subtitle{A Comparison of Models \& Methods}
\unisection{Faculty of Sciences}
\department{Department of Computer and Information Science}
\group{Modelling of Complex Self-Organizing Systems Group}
\supervisorOne{Dr. Stefano Tognazzi}
\supervisorTwo{Prof. Dr. Tatjana Petrov}

\headFoot{14}
\bibliography{resources} 

\begin{document}

\newgeometry{left=2.5cm, right=2.5cm, bottom=2cm, top=2.5cm, headheight=12pt, headsep=0.8cm, footskip=22pt}
\thesistitlepage[language=english]{M.Sc. Project Proposal}
\restoregeometry


\section{Introduction} Stochastic dynamical systems are present in many natural phenomenon. Different methods of modelling these yields different advantages when it comes to composition, minimization, executability, expressiveness, succintness and granularity. \\
The two predominant representations may be subdivided resulting in further distinct properties e.g. an automaton-oriented approach allows for degrees of freedom when defining the transition function, while a differential equivalence-based approach differs in the constraints and assumptions the concrete form can make. \\
 Chemical reaction networks~\autocite{mc_agg_crn}, quantum computing~\autocite{moore2000quantum} and brownian motion~\autocite{einstein1906theory} are just some examples where the first example may be described by ordinary differential equations (ODEs), stochastic process algebras, stochastic Petri nets, stochiometric equations, guarded commands and continuous-time Markov chains~\cite{wolf}, the second one using differential equations~\autocite{von2018mathematical}, automata and grammars~\autocite{moore2000quantum}, whereas the thrid example was proven to exist as phenomenologically described using stochastic differential equations~\autocite{einstein1906theory}. \\
 

\section{Problem Statement \& Objectives of Study}
The mentioned models may quickly reach the boarder to the computationally feasible, if one wants to extract exact results matching the real world. \\
Especially required are not only careful implementations --- no matter if it's numercial solvers, sampling techniques, simulations or even physical hardware --- but also appropriate minimization techniques with feasible computational complexity. \\
Another desired trait is to reduce the system further to only answer the question under consideration exactly, ignoring other aspects that are irrelevant to the examiner. \\
During the project and the subsequent thesis those methods and their properties shall be the subject of my studies. \\

Explicitly the following topics shall be investigated
\begin{itemize}
 \item automata-based \& differential equivalence-based modelling and the properties the different approaches have,
 \item minimization of both models without and with simplifying assumptions (e.g. only a specific information is relevant),
 \item connections and discords between the approaches, their properties and the results of minimizations of each under varying assumptions and conditions.
\end{itemize}
This induces the following goals:
\begin{enumerate}
 \item Implement the minimization algorithm for weighted automata proposed by \autocite{Kiefer2013OnTC}.
 \item Implement the minimization algorithm for ODEs proposed by \autocite{Cardelli2017MaximalAO}.
 \item Develop several reference examples serving as reproducible benchmarks in terms of qualitative and quantitative results regarding the properties of the models and the minimizations performable on each.
 \item Write a report including a formal description of the models, their properties, implications by refinements and extensions, connection \& discords, as well as comparisons of the results of the minimization.
\end{enumerate}


\section{Related Work}

\section{Methods}

\nocite{*}
\printbibliography

\appendix

\section{Timeline \& Milestones}

\end{document}
