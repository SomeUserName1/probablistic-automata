\documentclass[11pt, rgb]{scrartcl}
\usepackage{themeKonstanzDBIS} % Muss immer verwendet werden (Standardpaket)
\format{a4}

% Thesis information        %
\date{\today}
\year{2020}
\author{Fabian Klopfer}

\title{Minimization of Stochastic Dynamical Systems}
\subtitle{A Comparison of Models \& Methods}
\unisection{Faculty of Sciences}
\department{Department of Computer and Information Science}
\group{Modelling of Complex Self-Organizing Systems Group}
\supervisorOne{Dr. Stefano Tognazzi}
\supervisorTwo{Prof. Dr. Tatjana Petrov}

\bibliography{resources} 

\begin{document}

\newgeometry{left=2.5cm, right=2.5cm, bottom=2cm, top=2.5cm, headheight=12pt, headsep=0.8cm, footskip=22pt}
\thesistitlepage[language=english]{M.Sc. Project Proposal}
\restoregeometry


\section{Introduction} Stochastic dynamical systems are present in many natural phenomenon. Different methods of modelling these yields different advantages when it comes to composition, minimization, executability, expressiveness, succintness and granularity. The two predominant representations may be subdivided resulting in further distinct properties e.g. an automaton-oriented approach allows for degrees of freedom when defining the transition function, while a differential equivalence-based approach differs in the constraints and assumptions the concrete form can make.
 Chemical reaction networks~\autocite{mc_agg_crn}, quantum computing~\autocite{moore2000quantum} and brownian motion~\autocite{einstein1906theory} are just some examples where the first example may be described by ordinary differential equations (ODEs), stochastic process algebras, stochastic Petri nets, stochiometric equations, guarded commands and continuous-time Markov chains~\cite{wolf}, the second one using differential equations~\autocite{von2018mathematical}, automata and grammars~\autocite{moore2000quantum}, whereas the thrid example was proven to exist as phenomenologically described using stochastic differential equations~\autocite{einstein1906theory}. \\
 

\section{Problem Statement \& Objectives of Study}
The mentioned models may quickly reach the boarder to the computationally feasible, if one wants to extract exact results matching the real world. Especially required are not only careful implementations --- no matter if it's numercial solvers, sampling techniques, simulations or even physical hardware --- but also appropriate minimization techniques with feasible computational complexity. Another desired trait is to reduce the system further to only answer the question under consideration exactly, ignoring other aspects that are irrelevant to the examiner. During the project and the subsequent thesis those methods and their properties shall be the subject of my studies. Explicitly I want to investigate automata-based \& differential equivalence-based modelling and the properties the different approaches have, the minimization of both models without and with simplifying assumptions (e.g. only a specific information is relevant), the connections and discords between the approaches, their properties and the results of minimizations of each under varying assumptions and conditions.

This induces the following goals:
\begin{enumerate}
 \item Implement the minimization algorithm for weighted automata proposed by \autocite{Kiefer2013OnTC}.
 \item Implement the minimization algorithm for ODEs proposed by \autocite{Cardelli2017MaximalAO}.
 \item Develop several reference examples serving as reproducible benchmarks in terms of qualitative and quantitative results regarding the properties of the models and the minimizations performable on each.
 \item Write a report including a formal description of the models, their properties, implications by refinements and extensions, connection \& discords, as well as comparisons of the results of the minimization.
\end{enumerate}



\section{Methods \& Literature}
As project and thesis contain mainly two aspects namely self-directed learning and some sort of contribution, the related literature may be divided into two categories: First educational textbooks \& references containing general definitions, theorems and second application or research-oriented references \& papers containing theorems and algorithms on the topic where the contribution shall be made. For this scope the literature from the first category is on differential equations, weighted automata and additional topics that are relevant due to usage in the second category. The literature for the second category will be discussed in the next section. \\
\subsection{Differential Equations} (DE) is an equation relating functions with their derivatives. Those equations are called ordinary if the unknown functions are functions of one independent variable, while they're called partial if the function depends on more than one variable and contains partial derivatives. A differential equation is called linear if it is linear in the unkown function and non-linear if the equation is not expressable as linear in the unknown function. The coefficients of differential equations may be arbitrary, that is e.g. non-linear even when the differential equation is linear. Differential equations can also be homogenous or heterogenous. They're homogenous if all functions in the equation are of the same degree and homogenous, that is if you multiply all parameters of the function by a scaling factor, the value of the function is also multiplied by the scaling factor to the power of the function degree. If a DE is not homogenous it is called heterogenous. There are some extensions to the framework: Fractional differential equations generalize to non-integer derivatives, integro-differential equations is a generalization to relate funtions not only to their derivatives but also to integrals of themselves, functional differential equations allow to consider the derivatives with varying argument avlues (e.g. $x+1$, $x-3$ instead of $x$) what is widely used in electrical engineering e.g. derivatives a previous or future points in time~\autocite{richard}, differential-algebraic equations that allow defining constraints on the solution space by algebraic terms in the differential equation e.g. in electrodynamics when Kirchoffs law is to be preserved. This list is not complete not comprehensive, but finally there is one more important extension to DEs: Stochastic differential equations allow for modelling stochastic processes using differential equations with additional stochastic terms. As the field is rather broad, a lot of different references are required, which are mentioned here and listed in the bibliography:~\autocite{dae, ode_dae, dtsp, fde, handbook_de, ide, nlc, simmons_de, sde}. Minimization of DEs is a broad field that is an active research topic for several decades. An example of a current approach aggregates variables whose dynamics can explicilty be described when aggregated and variables with equivalent dynamics~\autocite{Cardelli2017MaximalAO}. \\

\subsection{Automata} are defined by Tuples of different length depending on the concrete subtype but always consist of a set of states, and a transition function. Those models differ in the type of memory, the transition and acceptance semantics. Automata define implicitly languages that follow certain construction rules called grammar. The expressiveness \& complexity of the grammar and the automaton model are directly coupled as the transitions and productions of automata and grammars are. Automata may either be deterministic or non-deterministic meaning that there is only one or many different choices per input when making a transition. If the environment decides what transition is taken (e.g. when making a typo and the spell checker marks it so that you correct it) it is called external non-determinism. If the automaton decides which transition is taken (e.g. when you make a typo, notice it by visual inspection and correct it) it's called internal non-determinism. Important is that for the same input character you have different transition and that either factors outside of the system or inside the system decide what transitions are to be taken. Finite State automata are memoryless and are equivalent to regular languages. Pushdown automata have a stack and are equivalent to context-free languages while Turing machines have unlimited, adressable can express context-sensitive and arbitrary languages. An extension to this basic framework are weighted automata, where each transition is assigned a number that specifies how likely it is to take the transition. A special case of that are probabilistic automata where certain transitions' weights have to fullfill the Kolmogorov axioms. The concrete rules and semantics depend on the model that is chosen and there is quite a zoo of models~\autocite{sokolova}. There are also probabilistic extensions of automata with memory but overall there is no clear hierarchy like in the non-probabilistic case. \autocite{sokolova} attempted to order the chaos a bit but only covered memoryless automata and ordered them by the generality of the transition function. The implications of different memory models in the probabilistic context has not been studied yet to the best of the authors knowledge. As the field of probabilistic automata is relatively new compared to the field of DEs --- Probabilistic Automata were first defined in 1956~\autocite{automatastudies} in contrast to DEs which were developed in the 17th century by Newton~\autocite{newton} --- there are less sophisticated textebooks to rely on so more papers are used. Namely the references are~\cite{sokolova, automatastudies, ppat, ata, hwa, paz, wolf, stoelinga, survey, patypes, pg, rpga, stochsysI, stochsysII, rabin}. Minimization techniques for automata often rely on linear algebra, that is minimization of the transition matrix combined with unreachable state elimination~\autocite{schutz}.

\section{Related Work on Minimization}


\nocite{*}
\printbibliography

\appendix

\section{Timeline \& Milestones}

\end{document}
