\documentclass{article} % For LaTeX2e
\usepackage{nips15submit_e,times}
\usepackage{hyperref}
\usepackage{url}
\usepackage[backend=biber,style=numeric,
citestyle=numeric]{biblatex}
\usepackage{amsthm}
\usepackage{amsmath}
\usepackage{amssymb}


\title{Probabilistic Automata: Semantics, Equivalence \& Minimization}


\author{
Fabian~Klopfer \\
Department of Computer Science\\
University of Konstanz\\
\texttt{fabian.klopfer@uni.kn} \\
\And
Tatjana~Petrov \\
Department of Computer Science \\
University of Konstanz \\
\texttt{tatjana.petrov@uni.kn} \\
\AND
Coauthor \\
Affiliation \\
Address \\
\texttt{email} \\
}

% The \author macro works with any number of authors. There are two commands
% used to separate the names and addresses of multiple authors: \And and \AND.
%
% Using \And between authors leaves it to \LaTeX{} to determine where to break
% the lines. Using \AND forces a linebreak at that point. So, if \LaTeX{}
% puts 3 of 4 authors names on the first line, and the last on the second
% line, try using \AND instead of \And before the third author name.

\newcommand{\fix}{\marginpar{FIX}}
\newcommand{\new}{\marginpar{NEW}}

\newtheorem{definition}{Definition}[section]
\newtheorem{theorem}{Theorem}[definition]
\newtheorem{corollary}{Corollary}[theorem]
\newtheorem*{remark}{Remark}

%\nipsfinalcopy % Uncomment for camera-ready version

\begin{document}


\maketitle

\begin{abstract}
TODO
\end{abstract}

\section{Introduction}\label{intro}



\section{Background}\label{bg}
    \subsection{Category Theory}
        
        
    \subsection{Probability Theory}
        \begin{definition}[Sample Space, Events, $\sigma$-Algebra, Measurable Space, Measurable Function]
            Let $\Omega$ be a set, namely the set of possible outcomes of a chance experiment, called \textbf{sample space}.
            
            Let $\mathcal{F} \subseteq \mathcal{P}(\Omega)$ with $\mathcal{P}$ the powerset, a set of subsets of the sample space, whose elements are called \textbf{events}.
            
            Let $\Omega$ be a sample space, $\mathcal{F}$ a set of events.
            $\mathcal{F}$ is called a \textbf{$\sigma$-algebra} over $\Omega$, if and only if
            \begin{itemize}
            \item the sample space is contained in the set of events,
            \[ \Omega \in \mathcal{F} \]
            \item the set of events is closed under complementation,
            \[ A \in \mathcal{F} \Rightarrow \Omega \setminus A \in \mathcal F \]
            \item and the set of events is closed under countable union:
            \[ \forall i \geq 0: A_i \in F \Rightarrow (\bigcup_{n \in \mathbb{N} A_n) \in \mathcal{F}} \]
            \end{itemize}
            
            The pair $\left( \Omega, \mathcal{F} \right)$ is called a \textbf{measurable space}.

            Let $\left( \Omega_1, \mathcal{F}_1 \right), \left( \Omega_2, \mathcal{F}_2 \right)$ measurable spaces. A funtion $f: \Omega_1 \rightarrow \Omega_2$ is called a \textbf{measurable function} if and only if for every $A \in \mathcal{F}_2$ the preimage of A under $f$ is in $\mathcal{F}_1$.
            \[ \forall A \in \mathcal{F}_2: f^{-1}(A) \in \mathcal{F}_1 \]
        \end{definition}
            
        \begin{definition}[Probability Space, Probability Measure, Discrete and Continuous Probability Space and Measure]


        \end{definition}

        \begin{definition}[Random Variable, Probability Distribution, Distribution Funtion and Cumulative Density Function]
        \end{definition}
        
        \begin{definition}[Stochastic Process, Bernouilli \& Binomial Process]
            
        \end{definition}

        \begin{definition}[Geometric Distribution]
            
        \end{definition}
        
        \begin{definition}[(Negative) Exponential Distr.]
         
        \end{definition}

        \begin{theorem}[Memoryless property]
            
        \end{theorem}
            
        \begin{definition}[Markov Property, Markov Process, Time Homogeneity]
         
        \end{definition}
 
 
    \subsection{Automata Theory}
            \begin{definition}[Transition System]
         
        \end{definition}
        
        \begin{definition}[Labelled Transition System]
         
        \end{definition}
        
        \begin{definition}[Path, Trace, Cylinder Sets, Prefix, Postfix]
            
        \end{definition}
    
        \begin{definition}[Probabilistic Automata, Initial Distribution, transition probability function, stochastic matrix, transition probability matrix]
            
        \end{definition}
        
        \begin{definition}[Markov Chains, Discrete-Time, Continuous-Time, Labelled, (MDP?, Generalized Stochastic Petri Nets?)]
            
        \end{definition}
        
        \begin{definition}[Bisimulation Relations, Strong, weak, exact, ordinary, Prob., Buchholz]
            
        \end{definition}
        
        

\subsection{Semantics, Equivalence \& Minimization}\label{complex}
\section{Semantics}
        \subsection{Parametrization and Initialization}
    
        \subsection{Trace Semantics}
            Execution as Sparse Matrix Multiplication, 
            (constrained) reachability (pr.), path/trace distributions,
            ergodicity, state residency time, Uniformization
            
        
        
        \subsection{Transient Semantics \& Cylinder Sets of Executions (incl. Reward/weighted semantics)}
        wrt. probabilities?!
            
            \begin{definition}[Transient probability distribution]
            \end{definition}
           
        
        \subsection{Threshold semantics}
            
        
        \subsection{Funtion Transient Semantics (?)}
        word functions instead of probability functions?
           
        
        \section{Equivalence \& Tanja's Draft Equivalences}
        
        \subsection{Trace semantics is decidable in P for all}
                paz p.36, Kiefer  , Tzeng, Schützenberger, Bollig \& Zeitoun, Kiefer WA should hold for PA when using sufficient eps (theoretically unclean), Doyen, 
            
        \subsection{Transient semantics is lower NP, upper EXP}
            As we need to construct the postfix space which is a cylinder set, transfers also to reward/weighted transient
            
        \subsection{Word Function-based}
            Same as above, just that lower bound is less; stop on first false; maybe eq of kiefer helps here
                
        
            
     
    \section{Minimization + wrt. Draft Eqs.}
        \begin{definition}[Lumping/lumpable]
         
        \end{definition}
        \subsection{Approaches}
            \subsubsection{Partition Refinement - Coalgebraic Approach}
            Proof: PA minimization is in P for uninitialized. \\
            Deiffel, Wißmann, Paz p.24ff (IntroToProbabilisticAutomata)
            for all in O(n log n) as partition refinement is minimal (see sources above) for uninitialized Automata. \\
        
            \subsubsection{Schützenberger's Construction $\pm$ Arnoldi Iteration with Housholder Reflectors \& }
            Show either that is also in P oder correct Kiefers runtime analysis.
            For uninitialized see kiefer; TODO figure out if kiefers reduction is flawed or if the runtime analysis of his algo is flawed.
            
        \subsection{Wrt. different semantics}
            \subsubsection{Decidability}
            
            \subsubsection{Complexity}


%% Figure Example
%\begin{figure}[h]
%\begin{center}
%%\framebox[4.0in]{$\;$}
%\fbox{\rule[-.5cm]{0cm}{4cm} 
% \includegraphics[width=0.8\linewidth]{myfile.pdf} 
% \rule[-.5cm]{4cm}{0cm}}
%\end{center}
%\caption{Sample figure caption.}
%\end{figure}
%% Table example
%\begin{table}[t]
%\caption{Sample table title}
%\label{sample-table}
%\begin{center}
%\begin{tabular}{ll}
%\multicolumn{1}{c}{\bf PART}  &\multicolumn{1}{c}{\bf DESCRIPTION}
%\\ \hline \\
%Dendrite         &Input terminal \\
%Axon             &Output terminal \\
%Soma             &Cell body (contains cell nucleus) \\
%\end{tabular}
%\end{center}
%\end{table}

\printbibliography

\end{document}
