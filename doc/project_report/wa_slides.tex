 \documentclass[rgb]{beamer}

\usetheme{Konstanz}
\format{169}

\title{Minimization of Stochastic Dynamical Systems:}
\titleCorporateDesign{Minimization of}{Stochastic Dynamical Systems}{A Comparison of}{Models \& Methods}
\author{Fabian Klopfer} 
\date{\thedate}
\institute{Modelling of Complex Self-Organizing Systems Group}

\bibliography{resources}
%\renewcommand*{\bibfont}{\tiny}

\begin{document}
    \usebeamerfont{normalfont}
    \begin{frame}
        \titlepage
    \end{frame}

    \begin{frame}{Motivation}
     \begin{itemize}
      \item Dynamical processes describe everything that changes
      \item DEs are the most common modelling tool in natural sciences
      \item Automata are the underlying computational model in current hardware and software systems
      \item Automata also describe dynamical processes
      \item Continuous time Markov chains e.g. for chemical reaction networks, quantum automata, $\dots$
     \end{itemize}
    \end{frame}
    
    \begin{frame}{Motivation}
        Differential Equations (DE) are omni-present in natural science:
        \begin{columns}
            \begin{column}{0.3\textwidth}
            Phisics \& Chemistry
                \begin{itemize}
                    \item electrodynamics
                    \item thermodynamics \& diffusion
                    \item fluid dynamics \& plasma physics
                    \item classical mechanics
                    \item celestial motion \& general relativity
                    \item nuclear physics \& quantum mechanics
                    \item Chemical reactions
                \end{itemize}
            \end{column}
            \begin{column}{0.3\textwidth}  %%<--- here
                Biology
                \begin{itemize}
                 \item Population dynamics
                 \item wound healing
                 \item epidemology
                \item blood flow
                \item reaction-diffusion equations
                \item Hodgkin Huxley model
                \item capiliary pressure
                 \end{itemize}
            \end{column}
            \begin{column}{0.3\textwidth}  %%<--- here
                Engineering \& Economics
                \begin{itemize}
                 \item optimal control
                 \item optimization techniques
                 \item electrial engineering
                \item signal processing
                \item trafic flow
                \item mean field game theory
                \item advertising effects
                \item economic growth
                \item price evolution
                 \end{itemize}
            \end{column}
        \end{columns}
    \end{frame}
    
    \begin{frame}{Why \alert{Stochastic} Dynamical Sytems?}
        Laplace's demon:
        \begin{center}
            \textit{An intellect which would know all forces that set nature in motion, and all positions of all items of which nature is composed, if this intellect were also vast enough to submit these data to analysis, it would embrace in a single formula the movements of the greatest bodies of the universe and those of the tiniest atom; for such an intellect nothing would be uncertain and the future just like the past would be present before its eyes.} \\
            - Pierre-Simon Laplace, \autocite{laplace1998pierre}
        \end{center}
        \vfill
        \begin{itemize}
            \item Only the universe itself exists since its' beginnings $\Rightarrow$ initial condition problem
            \item Thermodynamic and quantum mechanical irreversability $\Rightarrow$ Cannot certainly infer past
            \item Chaos Theory: Butterfly effect
        \end{itemize}
    \end{frame}
    
    \begin{frame}{Research Questions}
        \begin{enumerate}
            \item How to model stochastic dynamics universally and exactly?
            \item How do automata and DE models relate? 
            \item How do reduction, approximation and minimization methods relate?
            \item Is it possible to automatically combine bounded-error approximations, minimizations, reductions of order, $\dots$ to achieve minimal computational effort?
        \end{enumerate}
        \vfill
        \textbf{TL;DR:} What's the most exact, interpretable, computationally fastest and memory-greediest approach of modelling arbitrary stochastic dynamical systems and \alert{why} is that so?
    \end{frame}
    
    
    \begin{frame}{Example: Chemical Reaction Networks}
        Taken from \autocite{mc_agg_crn} with support of \autocite{van1992stochastic}. \\
        Let the chemical reaction network $CRN$ be a 2-tuple $(S,R)$ with 
        \[S = \{S_1, S_2, \dots, S_n\}\] the set of distinct species and 
        $R = \{R_1, R_2, \dots, R_k\}$ the set of reactions in the form 
        \[ a_{1j} S_1, a_{2j} S_2 \dots, a_{nj} S_n \xrightarrow{k_j}  a'_{1j} S_1, a'_{2j} S_2 \dots, a'_{nj} S_n \]
        where $a_j$ is called to consumption vector, $a'_j$ the production vector, $k_j$ the reaction rate contant and 
        \[ \nu_j = a'_j - a_j \]
        the change vector of $R_j$
    \end{frame}
  
    \begin{frame}{Classical chemical kinetics}
        \begin{itemize}
            \item Continuous, deterministic ODE-based model.
            \item Assumes constant pressure and temperature, well-stirredness.
            \item Based on concentrations rather then particles.
            % $[N] = \frac{|S_i|}{A \cdot V} \frac{mol}{dm^3}$ where $A = 6.23 \cdot 10^{23} \text{mol}^{-1}$ 
        \end{itemize}
        \vskip 1cm
        Let $s_0 = (s_1, s_2, \dots, s_n)$ be an initial state. The deterministic reaction rate is given by 
            \[ \lambda_j (s_i) = k_j \prod_{i=1}^n s_i^{a_{ij}} \]
        and the model by the solution of n coupled non-linear ODEs
            \[ \frac{d}{dt} s_i(t) = \sum_{j = 1}^{k} \nu_{ij} \lambda_j (s_i(t))  \]
        \vfill
        Problem: Can not explain variations present in reality
    \end{frame}
    
    \begin{frame}[allowframebreaks]{Stochastic chemical kinetics}
    \begin{itemize}
        \item Discrete, stochastic CTMC-based model.
        \item Assumption: Thermal equilibrium, well-stirred
        \item Based on particles.
        \item Transient probability distribution is measure of interest
        % $[N] = \frac{|S_i|}{A \cdot V} \frac{mol}{dm^3}$ where $A = 6.23 \cdot 10^{23} \text{mol}^{-1}$ 
    \end{itemize}
    \vfill
    Let $x_0 = (x_1, x_2, \dots, x_n)$ be an initial state. The stochastic reaction rate for each reaction is given by 
        \[ \tilde\lambda_j (x_) = c_j \prod_{i=1}^n \binom{x_i}{a_{ij}} \]
        
    \framebreak
        
    The model is def'ed by the CTMC ${X_t}$ over the states $$S = \{x | x \text{ reachable from } x_0 \}$$ with $p_0(x_0) = 1$ and the generator matirx
        \[ w(x,y) = \sum_{j = 1}^k \lambda_j(x) 1_{y = x - \nu_j}  \]

        
    Transient probability distributions can then be obtained by solving the chemical master equation (CME)
    \[ \frac{d}{dt} p^{(t)} (x) = \sum_{x-\nu_j \in S} \sum_{j = 1}^{k} \lambda_j (x - \nu_{j}) p{(t)}(x - \nu_{j}) - \sum_{j = 1}^k \lambda_j (x) p{(t)}(x) \]
    
    \begin{itemize}
     \item Instead of solving the CME, do dynamic MCMC approximation on the CTMC
     \item stochastic simulation algorithm (SSA)
    \end{itemize}
    \end{frame}
    
    \begin{frame}
        To summarize:
        \begin{itemize}
             \item classical chem. kin. use coupled ODEs to approx. stochastic model in the limit of species and volume to infinity at constant concentrations
             \item Continuous Markov Process (CMP) models stochastic chem. kin.
             \item solving Master Equation provides transient probability distribution
             \item which is in turn approx. by MCMC on the CTMC def'ed by the CMP
        \end{itemize}
        Questions:
        \begin{itemize}
         \item Are Monte Carlo methods on automata always cheaper than numerical solution of high dimension \& order DEs?
         \item Is there always an automaton that can approximate the DE?
         \item If reducting the complexity in either the CTMC (minimization) or the DE (reduction of order, fwd/bwd eq., QSSA), how does the counterpart change if existant?
         \item When only interested in certain quantities: How do we adapt the models?
        \end{itemize}

       
    \end{frame}

    
    \begin{frame}{Goals}
        \begin{enumerate}
            \item Implement minimization algorithm for weighted automata \autocite{Kiefer2013OnTC}.
            \item Implement model reduction algorithm for ODEs \autocite{Cardelli2017MaximalAO}.
            \item Develop reproducible benchmarks
            \item Write report including 
                \begin{itemize}
                 \item formal description of models, properties, specializations
                 \item connections \& discords of models
                 \item Description of dimensionality reductions, minimizations, approximations, reduction to partial models for each
                 \item compare reduction, minimization and approximation techniques
                 \item compare 1. \& 2. quantitively by results 3.
                \end{itemize}
        \end{enumerate}
    \end{frame}

    
    \begin{frame}{Bibliography}
        \printbibliography
    \end{frame}
\end{document}
