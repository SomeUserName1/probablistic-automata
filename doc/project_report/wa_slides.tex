 \documentclass[rgb]{beamer}
\usepackage{listings}

\usetheme{Konstanz}
\format{169}

\title{Minimization of Weighted Automata}
\titleCorporateDesign{Minimization}{of}{Weighted}{Automata}
\author{Fabian Klopfer}
\date{12.06.2020}
\institute{Modelling of Complex Self-Organizing Systems Group}

\bibliography{resources}
%\renewcommand*{\bibfont}{\tiny}

\begin{document}
    \usebeamerfont{normalfont}
    \begin{frame}
        \titlepage
    \end{frame}

    \begin{frame}{Introduction}
        In the last presentation $\dots$ \\ \vspace{0.5cm}
        \begin{itemize}
            \item two models for stochastic dynamical systems were considered: \\ \vspace{0.2cm} 
                Weighted Automata (WA) and Differential Equations (DE)\\ \vspace{0.8cm} 
            \item an example for modelling a CRN's dynamics in both models was given: \\
            \vspace{0.3cm} 
            \begin{itemize}
             \item[DE] Solving Chemical Master Equation 
             \item[WA] Monte Carlo CTMC
            \end{itemize}
        \end{itemize}
    \end{frame}
    \begin{frame}{Goals specified}
        \begin{enumerate}
            \item Implement minimization algorithm for weighted automata \autocite{Kiefer2013OnTC}. {\Huge $\checkmark$ }
            \item Implement model reduction algorithm for ODEs \autocite{Cardelli2017MaximalAO}.
            \item Develop reproducible benchmarks
            \item Write report including
        \end{enumerate}
    \end{frame}
    
    \begin{frame}{What has been done so far}
        \begin{itemize}
            \item Software Requirement Specification \& Software Design Document
            \item Random Basis Minimal WA Construction Algorithm by Kiefer/Schützenberger~\autocite{Kiefer2013OnTC}
            \item Execution of example by Matlab script and hand
            \item Implementation of minimization \& equivalence algorithm, interfaces, TUI, CLI, tests
        \end{itemize}
    \end{frame}
    
    \begin{frame}[allowframebreaks]{The Weighted Automaton Minimization Algorithm}
        Weighted Automaton $A = \left( n, \Sigma, \alpha, \mu, \eta \right)$, where
        \begin{itemize}
         \item $n$ the number of states
         \item $\Sigma$ the input alphabet
         \item $\alpha$ the initial vector with a non-zero value for all starting states
         \item $\mu$ the set of transition matrices, one per input character
         \item $\eta$ the final vector with non-zero values for all ending states
        \end{itemize}
        \vspace{1cm}
        Author claims $\mathcal{O}(log^2 n)$ runtime, but this is not correct as we will see later
        In the following slide we use the notion of a forward reduction, but the backwards reduction is analogous besides minor variations
        \framebreak
        
        \begin{itemize}
         \item Find a basis $F$ of the prefix space using random vectors $r_i$
         \begin{itemize}
          \item Add the vectors of all prefix words up to length n together and multiply this vector by n different factors yielding $\{v_1, \dots, v_n\}$
          \item Factors are derived by random vectors and structure of prefixes
          \item Base is then the maximally linear independent subset of $\{ \alpha, v_1, \dots, v_n\}$
         \end{itemize}
         \vspace{0.5cm}
         \item Use basis to do Schützenberger Construction~\autocite{schutz}: $\overrightarrow{A} = (\overrightarrow{n}, \Sigma, \overrightarrow{\alpha}, \overrightarrow{\mu}, \overrightarrow{\eta})$
            With
            \begin{itemize}
             \item $\overrightarrow{\mu} = \overrightarrow{F} \mu \overrightarrow{F}^{-1}$
             \item $\overrightarrow{\alpha} = e_1$
             \item $\overrightarrow{\eta} = \overrightarrow{F}\eta$
             \item $\overrightarrow{n} = \text{rank}(\overrightarrow{\mu})$
            \end{itemize}

        \end{itemize}
    \end{frame}

    \begin{frame}{The Weighted Automaton Minimization Algorithm: Pseudo Code}
        b%Pseudo code
    \end{frame}

    \begin{frame}{The Weighted Automaton Minimization Algorithm: Example}
        c%Pseudo code
    \end{frame}
    
    \begin{frame}{Implementation Details}
        d
    \end{frame}
    
    
    \begin{frame}{Up Next}
        e
    \end{frame}
  
    
    \begin{frame}{Bibliography}
        \printbibliography
    \end{frame}
\end{document}
