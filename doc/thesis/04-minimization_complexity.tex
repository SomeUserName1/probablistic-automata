\chapter{Semantics, Equivalence \& Minimization}\label{\positionnumber}  
        The general case, may help with some things, e.g. if proven independent of the semiring used. So for each subsection here starting with WAs independent of the Semiring that is used, continue with PA results that are as independent as possible from the concrete transition structure. Finally apply the aforementioned results to LMCs/MCs. Also discriminate between DTMC, MDP, PA model(s) I.e. per subsection apply the folling structure
        
            \paragraph{WA - General Case for arbitrary Semirings}
            
            \paragraph{PA - Results for other transition structures}
            
            \paragraph{MC-like models - Application of above and other literature on specific example of LMC}


    \section{Semantics}
        \subsection{Parametrization and Initialization}
    
        \subsection{Trace Semantics}
            Execution as Sparse Matrix Multiplication, 
            (constrained) reachability (pr.), path/trace distributions,
            ergodicity, state residency time, Uniformization
            
        
        
        \subsection{Transient Semantics (incl. Reward/weighted semantics)}
        wrt. probabilities?!
            
            \begin{definition}[Transient probability distribution]
            \end{definition}
                       
        
        \subsection{Threshold semantics}
            
        
        \subsection{Funtion Transient Semantics (?)}
        word functions instead of probability functions?
           
        
        
        %\paragraph{Connection to Systems of Equations}
        % ODEs as central limit of CTMCs & their SDEs
        %[Prof. Thomas G. Kurtz, wisc]
        %paragraph{Parametrized Automata as Stochastic Phase Space Approximaters}
        
    %\section{Languages \& Grammars}
    %    \paragraph{Threshold Language}
        
    %    \paragraph{Transient/Population Language}
        
        
    \section{Equivalence \& Tanja's Draft Equivalences}
            \begin{definition}[Bisimulation Relations, Strong, weak, forward, backward Prob., Buchholz]
            
        \end{definition}
        
        \subsection{Trace semantics is decidable in P for all}
                \begin{definition}[trace Equivalence, branching bisim.]
                 
                \end{definition}

                paz p.36, Kiefer  , Tzeng, Schützenberger, Bollig \& Zeitoun, Kiefer WA should hold for PA when using sufficient eps (theoretically unclean), Doyen, \\
                Let automata $A_i = \left( S_i, \Sigma_i, M^{(i)}, \pi_i, \eta_i \right)$ for $i \in \{1, 2\}$
                Equivalence of initialized automata:
                \[ A_1(\pi_1) ~ A_2(\pi_2) \Leftrightarrow \forall \sigma \in \Sigma: \ \pi_1 \eta_1 = \pi_2 \eta_2 \wedge \pi_1 M^{(1)}_{\sigma} = \pi_2 M^{(2)}_{\sigma}  \]
                How about uninitialized? \\
                
                Problem with proof in [1]: don't they construct the powerset (sum-sustituting recursively, i.e. for each letter in alphabet replace in last formula), negating this by ``there are only $n_1+n_2$ linear independent formulas with $n_1+n_2$ variables...?
                
        \subsection{Transient semantics}
        \begin{definition}
         transient equivalence, , finite-horizon bisimulation (bisimulation up to time-step k)
        \end{definition}

            Finite horizon bisim.
            What is T here? A Trace with a certain property (e.g. ending with $\sigma$ or up to $k$ steps?)
            \[ A_1(\pi_1) ~ A_2(\pi_2) \Leftrightarrow \forall \sigma \in \Sigma: \ \pi_1 M^{(1)}_{\sigma} \eta_1 = \pi_2 M^{(2)}_{\sigma} \eta_2 \wedge \pi_1 T_1 = \pi_2  T_2 \]
            
        \subsection{Word Function-based}
            finite-horizon + reward fn
            \[ A_1(\pi_1) ~ A_2(\pi_2) \Leftrightarrow \forall \sigma \in \Sigma: \ \pi_1 M^{(1)}_{\sigma} \eta_1 \mu_{\sigma} = \pi_2 M^{(2)}_{\sigma} \eta_2 \mu_{\sigma} \wedge \pi_1 T_1 = \pi_2  T_2 \]    
        
            
     
    \section{Minimization + wrt. Draft Eqs.}
        \subsection{Approaches}
            \subsubsection{Partition Refinement - Coalgebraic Approach}
            Proof: PA minimization is in P for uninitialized/strict lumpability \\
            Deiffel, Wißmann, Paz p.24ff (IntroToProbabilisticAutomata)
            for all in O(n log n) as partition refinement is minimal (see sources above) for uninitialized Automata. \\
            Rubino sericola \\
            in case ODEs: Tribastones work \& boreale
        
            \subsubsection{Schützenberger's Construction and Arnoldi Iteration with Housholder Reflectors \& }
            Show either that is also in P oder correct Kiefers runtime analysis.
            For initialized see kiefer; 
            
        \subsection{Wrt. different semantics}
            \subsubsection{Decidability}
            trace semantics: is decidable, see e.g. Kiefer or Mateus, Qiu, Li;;;; Trace equivalence PSpace Complete (according to TU Eindhoven's j.f. Groote \\
            Transient semantics ? finite horizon bisimulation minimization using partition refinement \\
            weighted transient semantics? \\
            
            \subsubsection{Complexity}
            trace semantics: TODO figure out if kiefers reduction is flawed or if the runtime analysis of his algo is flawed. \\
            Transient semantics? n log m ?\\
            weighted transient semantics? \\
            
    
    %\section{Learning \& Parameter Synthesis}
